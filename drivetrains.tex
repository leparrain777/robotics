\documentclass[12pt]{article}
\usepackage{amsmath}
\usepackage{amsfonts}
\usepackage[left=1in,right=1in]{geometry}
\usepackage[utf8]{inputenc}
\usepackage[english]{babel}
\usepackage{multicol}
\begin{document}


\section{Tank Drive}
\subsection{Cost}
\begin{itemize}
    \item Needs only 2 gearboxes. (4 if designed for mecanum) (upgraded if 6 motors used)
    \item Motors used is up to the year, with options for 2 or 4. (or 6 with upgraded gearboxes)
    \item Wheels used is 6. ( or 4 if designed for meccanum)
    \item Can use any size of wheel if chains are used. Can use 6"+ if belts are used.
    \item Can use any type of wheel. (2 normal and 2 omni used for rotational axis change)
    \item Can use chain or belts with sprockets or pullies respectively.
\end{itemize}
\subsection{Programming}
\begin{itemize}
    \item Can be done by freshmen with some help.
    \item Extremely straightforeward.
    \item Short development time.
    \item Easiest to program for autonomous. Dead reckoning or encoder reading possible.
    \item No need for a relative heading from a gyro.
\end{itemize}
\subsection{Capabilities}
\begin{itemize}
    \item Tied for best design for pushing in the direction of the wheels.
    \item Resists perpendicular pushing perfectly.
    \item Resists $Sin(|\Theta|)$ in general where $\Theta$ is the angle of the pushing force from perpendicular.
    \item Can rotate about anywhere on the axis of the center wheels.
    \item Single best option for traction reliability. (track optional to help traction if needed)
    \item One of the best options for max speed/acceleration as all motors are alligned.
    \item More compact and leaves more space for interior components.
\end{itemize}
\subsection{Hinderances}
\begin{itemize}
    \item Must rotate to move in another direction.
\end{itemize}
\subsection{Driver Usage}
\begin{itemize}
    \item Easiest to learn.
    \item No orientation required, no gyro/accelerometer drift.
    \item Easy to build and most likely part of a practive bot so easy to practice on.
    \item Easiest control scheme, only needs 2 axis to control on seperate hands leading to less human error.
    \item More likely to have experienced drivers.
\end{itemize}
\subsection{Design and Build}
\begin{itemize}
    \item Easiest to design.
    \item Easiest to build.
\end{itemize}
\subsection{Assembly and Repair}
\begin{itemize}
    \item Easiest to assemble if designed well.
    \item Easy but annoying to repair with chain and sprockets.
    \item Either easy or very hard to repair with belts. (spares required)
\end{itemize}
%%%%%%%%%%%%%%%%%%%%%%%%%%%%%%%%%%%%%%%%%%%%%%%%%%%%%%%%%%%%%%%%%%%%%%%%%%%%%%%%%%%%
\section{Mecanum Drive}
\subsection{Cost}
\begin{itemize}
    \item Needs 4 gearboxes.
    \item Motors used is 4.
    \item Wheels used is 4.
    \item Can use any size of mecanum wheel if chains are used.
    \item Needs a full matching set to function, with 2 of each chirality.
    \item Can use chain or belts with sprockets or pullies respectively.
\end{itemize}
\subsection{Programming}
\begin{itemize}
    \item Can be done by freshmen with a some help to a lot of help.
    \item Programming requires knowlege of how wheels are oriented.
    \item Medium development time.
    \item Hardest to program for autonomous. Dead reckoning isn't even a guarantee.
    \item Can do without a gyro, but one is often used and is the source of problems.
\end{itemize}
\subsection{Capabilities}
\begin{itemize}
    \item Not great for pushing in the direction of the wheels, but it can do it.
    \item Resists roughly half of all pushing in any direction.
    \item Rotations possible, but not reliable due to rollers.
    \item Some conditional traction reliability. 
    \item Ideal for very small perpendicular to wheel motions.
    \item More compact and leaves more space for interior components.
\end{itemize}
\subsection{Hinderances}
\begin{itemize}
    \item Easy for most robots to push around.
    \item Hard to operate on uneven or slick terrain, with potential to get stuck.
    \item Wheels are prone to heavy wear and tear and flat spots causing sliding or skipping.
\end{itemize}
\subsection{Driver Usage}
\begin{itemize}
    \item Easy to learn. Very hard to master.
    \item Possible orientation corrections based on presence of gyro/accelerometer and its drift.
    \item Relatively easy to build but there may or may not be a practice bot ready.
    \item Harder control scheme, often uses at least 3 axis on the same joystick where human error is likely.
    \item Moderately likely to have somewhat experienced drivers, but very unlikely to have mastery.
\end{itemize}
\subsection{Design and Build}
\begin{itemize}
    \item Relatively simple to design.
    \item Needs space for extra gearboxes and power transmission.
    \item Easy to build.
\end{itemize}
\subsection{Assembly and Repair}
\begin{itemize}
    \item Easy to assemble but requires thought of wheel orientations.
    \item Somewhat harder to repair.
    \item Full set of spares required at all times.
\end{itemize}

\end{document}